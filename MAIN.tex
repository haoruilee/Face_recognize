\documentclass{article}

% if you need to pass options to natbib, use, e.g.:
%     \PassOptionsToPackage{numbers, compress}{natbib}
% before loading neurips_2020

% ready for submission
% \usepackage{neurips_2020}

% to compile a preprint version, e.g., for submission to arXiv, add add the
% [preprint] option:
%     \usepackage[preprint]{neurips_2020}

% to compile a camera-ready version, add the [final] option, e.g.:
%     \usepackage[final]{neurips_2020}

% to avoid loading the natbib package, add option nonatbib:
\usepackage[nonatbib]{neurips_2020}

\usepackage[utf8]{inputenc} % allow utf-8 input
\usepackage[T1]{fontenc}    % use 8-bit T1 fonts
\usepackage{hyperref}       % hyperlinks
\usepackage{url}            % simple URL typesetting
\usepackage{booktabs}       % professional-quality tables
\usepackage{amsfonts}       % blackboard math symbols
\usepackage{nicefrac}       % compact symbols for 1/2, etc.
\usepackage{microtype}      % microtypography


%%%%我的包
\usepackage{graphicx}
%\graphicspath{ {./images/} }
\usepackage{float} 
\title{Report of face-recognition by finetuning \\ ResNet and Haorui-Net}
\documentclass{article}
\usepackage[utf8]{inputenc}
\usepackage{booktabs} %三线表需要加载宏包{booktabs}
\usepackage{diagbox}
\usepackage{multirow}
\usepackage{listings}
\usepackage{xcolor}

%New colors defined below
\definecolor{codegreen}{rgb}{0,0.6,0}
\definecolor{codegray}{rgb}{0.5,0.5,0.5}
\definecolor{codepurple}{rgb}{0.58,0,0.82}
\definecolor{backcolour}{rgb}{0.95,0.95,0.92}

%Code listing style named "mystyle"
\lstdefinestyle{mystyle}{
  backgroundcolor=\color{backcolour},   commentstyle=\color{codegreen},
  keywordstyle=\color{magenta},
  numberstyle=\tiny\color{codegray},
  stringstyle=\color{codepurple},
  basicstyle=\ttfamily\footnotesize,
  breakatwhitespace=false,         
  breaklines=true,                 
  captionpos=b,                    
  keepspaces=true,                 
  numbers=left,                    
  numbersep=5pt,                  
  showspaces=false,                
  showstringspaces=false,
  showtabs=false,                  
  tabsize=2
}

%"mystyle" code listing set
\lstset{style=mystyle}



% The \author macro works with any number of authors. There are two commands
% used to separate the names and addresses of multiple authors: \And and \AND.
%
% Using \And between authors leaves it to LaTeX to determine where to break the
% lines. Using \AND forces a line break at that point. So, if LaTeX puts 3 of 4
% authors names on the first line, and the last on the second line, try using
% \AND instead of \And before the third author name.

\author{
   Haorui Li\thanks \\
  Chien-Shiung Wu College\\
  Southeast University\\
  % examples of more authors
  % \And
  % Coauthor \\
  % Affiliation \\
  % Address \\
  % \texttt{email} \\
  % \AND
  % Coauthor \\
  % Affiliation \\
  % Address \\
  % \texttt{email} \\
  % \And
  % Coauthor \\
  % Affiliation \\
  % Address \\
  % \texttt{email} \\
  % \And
  % Coauthor \\
  % Affiliation \\
  % Address \\
  % \texttt{email} \\
}

\begin{document}

\maketitle

\begin{abstract}
For face recognition, first, I use MTCNN and face.evoLVe for automatic data cleansing and change parameters in MTCNN to avoid dirty data. Then I trained two models, one is self-modified Resnet called Haorui-Net which use Cov2d layers in ResNet for fracture extraction and use pooling and softmax layers to do classifications, another is InceptionResNetV1 with pre-trained weight, and fine-tuning the model on classmates' data. Finally the best model recognizes 86/104 classmates in 48s. During the training process, I compare several different optimizers and combination of batch and epoch and use the best one.
\end{abstract}

\section{Data prepare}
\subsection{Face alignment}
To begin with, I use MTCNN[1] and \textit{face.evoLVe.PyTorch} for automatic face alignment.

MTCNN propose a deep cascaded multi-task framework which
exploits the inherent correlation between them to boost up Resnet's performance on face alignment, the architecture is as follows:
\begin{figure}[H]%H为当前位置,!htb为忽略美学标准,htbp为浮动图形
  \centering
  \caption{MTCNN's architecture}
%可选参数中width=\columnwidth选取了当前列宽
  \includegraphics[width=\columnwidth]{IMG/MTCNN.png} %插入图片,[]中设置图片大小,{}中是图片文件名
  \label{Fig.RNN} %用于文内引用的标签
\end{figure}
But I find though MTCNN is very fast, but it sometimes go wrong and bring in dirty data, like the Figure2, and these dirty data will definitely bring catastrophe for model trainning.
\begin{figure}[H]%H为当前位置,!htb为忽略美学标准,htbp为浮动图形
  \centering
  \caption{Samples of dirty data by MTCNN}
%可选参数中width=\columnwidth选取了当前列宽
  \includegraphics[width=\columnwidth]{IMG/ML大作业筛选展示.png} %插入图片,[]中设置图片大小,{}中是图片文件名
  \label{Fig.RNN} %用于文内引用的标签
\end{figure}
So I turn to \textit{face.evoLVe's face-align tools} and finally get good data.
This tool can be find at:
\begin{center}
  \url{https://github.com/ZhaoJ9014/face.evoLVe.PyTorch}
\end{center}
This tool is about 4-times slower than MTCNN, but brings no dirty data.

But I am wandering why MTCNN get these wrong results, because it is almost at state-of-the-art. And the face.evoLVe tool is designed base on MTCNN. So I test several parameters, It shows that when the default minim-window-size is undefined, mtcnn starts from 10x10 and tends to get wrong faces. So after I set the minimum size at 40x40, all results are good. 

\subsection{Rebuild folder architecture}
For quick detect image labels, I use \textit{torchvision.datasets.ImageFolder} to automatically read classmates name. To use this function, I rebuild the data folder's architecture by code.

Exactly, I use os.rename and string.split. Following are some codes I use to split the student number:

\begin{lstlisting}[language=Python, caption=Change folder names for ImageFloder function]
def replaceDirName(rootDir):
  #Change the folders' name under rootDir, split the student number by '-' or '_'
    num = 0
    dirs = os.listdir(rootDir)
    for dir in dirs:
        print('oldname is:' + dir)
        num = num +1
        try:
          temp = dir.split('_')[1]
        except IndexError:
          try:
            temp=dir.split('-')[1]
          except:
            print("This is not Number-Name structure", dir)
            continue
        except:
          print("This is not - or _ structure", dir)
          continue
        print('new name:',temp)
        oldname = os.path.join(rootDir, dir)
        newname = os.path.join(rootDir, temp)
        os.rename(oldname, newname)#replace
replaceDirName('align_data')
\end{lstlisting}




After rebuild the folder architecture, \textit{torchvision.datasets.ImageFolder} is able to automatically read sub-folders' name as image label. 

\subsection{Transforms}
After clean the data and align all the faces, I made some extra preparations for models robustness and these work has brought about 3-point increase in test accuracy.

When load in the data I perform some random transforms to the images to improve training. Different transforms can be attempted and I tried various ones, like Random-Color-Jitter and Random-Rotation, along with Random-Horizontal-Flip.
\begin{figure}[H]%H为当前位置,!htb为忽略美学标准,htbp为浮动图形
  \centering
  \caption{Examples of random Color Jitter}
%可选参数中width=\columnwidth选取了当前列宽
  \includegraphics[width=\columnwidth]{IMG/随机彩色.png} %插入图片,[]中设置图片大小,{}中是图片文件名
  \label{Fig.RNN} %用于文内引用的标签
\end{figure}
Finally I choose all these transforms to improve the model's robustness. And the random-color-jitter improves about 2 points in accuracy probably because classmates take photo at different light environment.

\section{Design model architecture}
Due to the fact that the data we have is small scale, it will be hard to train a model without over-fitting. So I think it is recognized to use some pre-trained model and do the fine-tuning.  What I have to do is design the final layers.

\subsection{Pre-trained ResNet}
The pre-trained weight I download is the Facenet trained by Google. They use triple loss and finally get 0.997 accuracy at Lwf, the High-Level modal structure of Facenet is as follow[2]:
\begin{figure}[H]%H为当前位置,!htb为忽略美学标准,htbp为浮动图形
  \centering
  \caption{High Level Modal Structure of Facenet}
%可选参数中width=\columnwidth选取了当前列宽
  \includegraphics[width=\columnwidth]{IMG/facenet.png} %插入图片,[]中设置图片大小,{}中是图片文件名
  \label{Fig.RNN} %用于文内引用的标签
\end{figure}
And for the first model, I use Inception-ResNet[3] to fine-tuning the model, which is designed for fine-tuning Facenet. The architecture of Inception-ResNet is as follow:
\begin{figure}[H]%H为当前位置,!htb为忽略美学标准,htbp为浮动图形
  \centering
  \caption{Inception-ResNet}
%可选参数中width=\columnwidth选取了当前列宽
  \includegraphics[width=18ex]{IMG/RESNET.png} %插入图片,[]中设置图片大小,{}中是图片文件名
  \label{Fig.RNN} %用于文内引用的标签
\end{figure}
The code of final layers are:
\begin{lstlisting}[language=Python, caption=Final layer Codes]
        self.block8 = Block8(noReLU=True)
        self.avgpool_1a = nn.AdaptiveAvgPool2d(1)
        self.dropout = nn.Dropout(dropout_prob)
        self.last_linear = nn.Linear(1792, 512, bias=False)
        self.last_bn = nn.BatchNorm1d(512, eps=0.001, momentum=0.1, affine=True)
        self.logits = nn.Linear(512, tmp_classes)
\end{lstlisting}
And I will modified the final layers, then test which model is the best.

\subsection{Modified ResNet}
From the upper section we can see the final six layers are:
\begin{lstlisting}[language=Python, caption=Final layers]
[Block8(
   (branch0): BasicConv2d(
     (conv): Conv2d(1792, 192, kernel_size=(1, 1), stride=(1, 1), bias=False)
     (bn): BatchNorm2d(192, eps=0.001, momentum=0.1, affine=True, track_running_stats=True)
     (relu): ReLU()
   )
   (branch1): Sequential(
     (0): BasicConv2d(
       (conv): Conv2d(1792, 192, kernel_size=(1, 1), stride=(1, 1), bias=False)
       (bn): BatchNorm2d(192, eps=0.001, momentum=0.1, affine=True, track_running_stats=True)
       (relu): ReLU()
     )
     (1): BasicConv2d(
       (conv): Conv2d(192, 192, kernel_size=(1, 3), stride=(1, 1), padding=(0, 1), bias=False)
       (bn): BatchNorm2d(192, eps=0.001, momentum=0.1, affine=True, track_running_stats=True)
       (relu): ReLU()
     )
     (2): BasicConv2d(
       (conv): Conv2d(192, 192, kernel_size=(3, 1), stride=(1, 1), padding=(1, 0), bias=False)
       (bn): BatchNorm2d(192, eps=0.001, momentum=0.1, affine=True, track_running_stats=True)
       (relu): ReLU()
     )
   )
   (conv2d): Conv2d(384, 1792, kernel_size=(1, 1), stride=(1, 1))
 ),
 AdaptiveAvgPool2d(output_size=1),
 Linear(in_features=1792, out_features=512, bias=False),
 BatchNorm1d(512, eps=0.001, momentum=0.1, affine=True, track_running_stats=True),
 Linear(in_features=512, out_features=8631, bias=True),
 Softmax(dim=1)]
\end{lstlisting}
Because earlier layers as containing the base-level information needed to recognize face attributes and base level characteristics, so I want to cut the layers after Conv2d and use some my own code, and just updating the final layers to include another 104 faces.

Put all beginning layers in an nn.Sequential:
\begin{lstlisting}[language=Python, caption=Keep the conv2d layers]
model_ft = nn.Sequential(*list(model_ft.children())[:-5])
\end{lstlisting}
Now, model modified is a torch model but without the final linear, pooling, batchnorm, and sigmoid layers.

After this, I design another final layers class includes sample Flatten and Normalize layers in a gesture to use features extracted by Cov2d layers, the codes are:
\begin{lstlisting}[language=Python, caption=Haorui Net]
#Change the final layers as follows
model_modified.avgpool_1a = nn.AdaptiveAvgPool2d(output_size=1)
model_modified.last_linear = nn.Sequential(
    Flatten(),
    nn.Linear(in_features=1792, out_features=512, bias=False),
    normalize()
)
model_modified.logits = nn.Linear(layer_list[4].in_features,104)
model_modified.softmax = nn.Softmax(dim=1)
model_modified = model_modified.to(device)
\end{lstlisting}
So the architecture is:
\begin{figure}[H]%H为当前位置,!htb为忽略美学标准,htbp为浮动图形
  \centering
  \caption{Haorui-Net Architecture}
%可选参数中width=\columnwidth选取了当前列宽
  \includegraphics[width=\columnwidth]{IMG/haoruinet.png} %插入图片,[]中设置图片大小,{}中是图片文件名
  \label{Fig.RNN} %用于文内引用的标签
\end{figure}

We can name it Haorui-Net. In the next section I will train these two models and show some details to pick the winner.

\section{Training and preferences}
After design the model, I begin the training step. Tried different epoch, batch size, learning rate and models.

\subsection{Check GPU Memory}
The options of batch size are often limited by GPU memory.

On my machine, I have a  single Tesla-P-100 with 16280 MiB memory, which means I have more choice on batch size and epochs. 

Use '!nvidia-smi' I get the following in formations of GPU memeory, it shows that 6869 MiB memory is located at device and I still have space to test. 
\begin{figure}[H]%H为当前位置,!htb为忽略美学标准,htbp为浮动图形
  \centering
  \caption{24 Epochs and 64 Batch-size}
%可选参数中width=\columnwidth选取了当前列宽
  \includegraphics[width=\columnwidth]{IMG/ML实验GPU.png} %插入图片,[]中设置图片大小,{}中是图片文件名
  \label{Fig.RNN} %用于文内引用的标签
\end{figure}
\subsection{Should I use Adam?}
Optimizer plays an important role in deep-learning, and different optimizer can have totally performance.

As we all know, "Adam" is honoured as an excellent optimizer, but should I use it too in my work? So I test another theoretically-good optimizer which is called RMS-prop, and the results in Tensorborad-X are as follows:
\begin{figure}[H]%H为当前位置,!htb为忽略美学标准,htbp为浮动图形
  \centering
  \caption{Trainning loss of RMS in TensorboradX}
%可选参数中width=\columnwidth选取了当前列宽
  \includegraphics[width=40ex]{IMG/loss_RMSprop.png} %插入图片,[]中设置图片大小,{}中是图片文件名
  \label{Fig.RNN} %用于文内引用的标签
\end{figure}
It shows that the loss of RMS optimizer finally  convergences at about 4.5, and in the preliminary stage it really decreased fast.

But with the same epochs and batch-size, which is 32 and 128, the Adam optimizer performs really better:

\begin{figure}[H]%H为当前位置,!htb为忽略美学标准,htbp为浮动图形
  \centering
  \caption{Trainning loss of Adam in TensorboradX}
%可选参数中width=\columnwidth选取了当前列宽
  \includegraphics[width=40ex]{IMG/loss_Adam.png} %插入图片,[]中设置图片大小,{}中是图片文件名
  \label{Fig.RNN} %用于文内引用的标签
\end{figure}
It shows that the loss of Adam optimizer finally  convergences at about 0.2, even though in the preliminary stage it decrease slower than RMS but finally it convergences at a better point. 

I also test the FPS of training and testing, but it shows that this two optimizer are almost the same:
\begin{figure}[htbp]
\centering
\begin{minipage}[t]{0.48\textwidth}
\centering
\includegraphics[width=6cm]{IMG/RMS_FPS.png}
\caption{FPS of RMS}
\end{minipage}
\begin{minipage}[t]{0.48\textwidth}
\centering
\includegraphics[width=6cm]{IMG/FPS.png}
\caption{FPS of Adam}
\end{minipage}
\end{figure}


As its shown above, Adam optimizer performs better and I will use it in trainning my model.
\subsection{Epochs and batch-size}
After choose several combinations of epochs and batch size,  I get the results as follows on Inception-ResNet:

\begin{table}[H]
\centering
\caption{Records of combination for ResNet}\label{tab:aStrangeTable}%添加标题 设置标签
\begin{tabular}{cccc}
\toprule
Epochs & Batch size& TP &Train FPS\\
\midrule
10& 16 & 21 & 427.4\\
24& 16& 26 & 420.7\\
24 & 32 & 41 & 279.6\\
24 & 64 & 75 & 153.4\\
32 & 64 & 71 & 161.5\\
24 & 128 & 80 & 149.5\\
32 & 128 & 77 & 233.9\\
24 & 256 & 70  &183.3\\
32 & 256 & 77  &192.8\\
64 & 256 & 76  &155.3\\
\bottomrule
\end{tabular}

%\caption{这是一张三线表}\label{tab:aStrangeTable}  标题放在这里也是可以的
\end{table}
From the chart we can see, more batch size often means better performance, but with more batch size, sometimes it need more epochs to minimize the loss, just like 256 batch size performs weaker than 128 batch size in 24 epochs, and become better in 32 epochs.

So finally, the ResNet performs its best at 24 epochs, 128 batch size and reaches 82 true positive. This model was saved as  '24-epoch-128bz-VGGFACE2-TEST80ACC.pb'.

With the chart above, I can qiuckly choose some combinations for Haorui-Net, and the results are as follows:

\begin{table}[H]
\centering
\caption{Records of combination for Haorui-Net}\label{tab:aStrangeTable}%添加标题 设置标签
\begin{tabular}{cccc}
\toprule
Epochs & Batch size& TP &Train FPS\\
\midrule
24 & 64 & 71 & 153.9\\
24 & 128 & 82 & 171.4\\
32 & 128 & 86 & 255.5\\
32 & 256 & 77 & 210.4\\
64 & 256 & 77 & 195.7\\
\bottomrule
\end{tabular}
\end{table}

Luckily, the Haorui-Net performs better than ResNet its best at 24 epochs, 128 batch size and reaches 82 true positive. This model was saved as  '32-epoch-128bz-MODIFIED-TEST86ACC.pb'.

So I'm proud to announce that Haorui-Net becomes the winner in this combination, with ten more ture-positive!

But what I want to point out is that, Haorui-Net is weaker in the decrease of loss, for ResNet, the minimum of loss is about 0.27 while training, but for Haorui-Net, the minimum loss is about 3.8, it probably means ResNet is designed more smarter in track and reduce the loss.

\section{Test and Conclusion}
Because in the training stage I use Face.LVe to process face images, now when test, using this tool will be slow, so I turn to MTCNN and by change its parameters it seldom detect wrong images.
\begin{lstlisting}[language=Python, caption=MTCNN Parameter]
mtcnn = MTCNN(image_size=160, 
        margin=0,
        min_face_size=60,
        thresholds=[0.6,0.7,0.7],
        factor=0.709,post_process=True,device=device)
\end{lstlisting}
I load the best model of Haorui-Net and the test of Face-Recognize shows:
\begin{figure}[H]%H为当前位置,!htb为忽略美学标准,htbp为浮动图形
  \centering
  \caption{Face Recognize Test}
%可选参数中width=\columnwidth选取了当前列宽
  \includegraphics[width=50ex]{IMG/新人脸识别测试.png} %插入图片,[]中设置图片大小,{}中是图片文件名
  \label{Fig.RNN} %用于文内引用的标签
\end{figure}
It takes about 0.46 second per student for face recognize and the accuracy is 82.7$\%$ for the best model of "Haorui Net", not so bad.

But this result is slower than ResNet:
\begin{figure}[H]%H为当前位置,!htb为忽略美学标准,htbp为浮动图形
  \centering
  \caption{Face Recognize Test}
%可选参数中width=\columnwidth选取了当前列宽
  \includegraphics[width=50ex]{IMG/人脸识别测试2.png} %插入图片,[]中设置图片大小,{}中是图片文件名
  \label{Fig.RNN} %用于文内引用的标签
\end{figure}



For Face-Verification, I find that it takes too long to run the function because it have to check all the faces, so I just check the first 40 faces and get the results below:
\begin{figure}[H]%H为当前位置,!htb为忽略美学标准,htbp为浮动图形
  \centering
  \caption{Face Verification Test}
%可选参数中width=\columnwidth选取了当前列宽
  \includegraphics[width=40ex]{IMG/人脸认证.png} %插入图片,[]中设置图片大小,{}中是图片文件名
  \label{Fig.RNN} %用于文内引用的标签
\end{figure}

In conclusion, I test the Resnet and hand-modified Haorui-Net, all based on pretrained weights, finally Haorui-Net win the competition in accuracy. I use Adam optimizer because it performs best in minimising loss. For the best model, it takes about 0.46 second per student for face recognize and the accuracy is 82.7 $\%$.

\section{Expectations}
Though my model get a good result in accuracy, but there still remains something I want to explore.

For example, my face-verification function runs too slow to verified all pictures and names, I think it perhaps due to my  algorithm is O($n^2$) and I write too many works to move data between GPU and CPU which is time-consuming. And I think perhaps use B+ tree or some other data structure can speed up the searching process, also, keep all the data on one device can avoid moving them.

Moreover, though my model works great on our classmate-dataset, but for actual industrial demand, sometimes the faces in picture is really small, slant, and only have side faces, like  surveillance videos. To recognize faces in these scenes, perhaps we have to made a 3D-model for faces[4], and use more skills to avoid overfitting like knowledge-distillation.[5]

In conclusion, there are still large space to modify this work for specific context.
%%%%%%%%%%%%%%%%%%%%%%%%%%%%%%%%
%%%%%%%%%%%%%%%%%%%%%%%%%%%%%%%%
%%%%%%%%%%%%%%%%%%%%%%%%%%%%%%%%

\section*{References}


\medskip

\small

[1] Zhang, K., Zhang, Z., Li, Z. & Qiao, Y. (2016). Joint Face Detection and Alignment using Multi-task Cascaded Convolutional Networks.. CoRR, abs/1604.02878.

[2]Schroff, F., Kalenichenko, D. & Philbin, J. (2015). FaceNet: A Unified Embedding for Face Recognition and Clustering (cite arxiv:1503.03832Comment: Also published, in Proceedings of the IEEE Computer Society Conference on Computer Vision and Pattern Recognition 2015)

[3]Szegedy, C., Ioffe, S., Vanhoucke, V. & Alemi, A. A. (2017). Inception-v4, Inception-ResNet and the Impact of Residual Connections on Learning. Proceedings of the 31st AAAI Conference on Artificial Intelligence (p./pp. 4278--4284), : AAAI Press.

[4]Dou, P., Zhang, L., Wu, Y., Shah, S. K. & Kakadiaris, I. A. (2015). Pose-robust face signature for multi-view face recognition.. BTAS (p./pp. 1-8), : IEEE. ISBN: 978-1-4799-8776-4


[5]Luo, P., Zhu, Z., Liu, Z., Wang, X. & Tang, X. (2016). Face Model Compression by Distilling Knowledge from Neurons.. In D. Schuurmans & M. P. Wellman (eds.), AAAI (p./pp. 3560-3566), : AAAI Press. ISBN: 978-1-57735-760-5



\end{document}
